\documentclass{beamer}
\usepackage[utf8]{inputenc}
%\usepackage[a4paper,margin=1.1in]{geometry}
%\usepackage[colorlinks,citecolor=magenta,linkcolor=black]{hyperref}
%\pdfpagewidth=\paperwidth \pdfpageheight=\paperheight
%\usepackage{amsfonts,amssymb,amsthm,amsmath,eucal,tabu,url}
\usepackage{amsmath}
\usepackage{pgf}
 \usepackage{array}
 \usepackage{pstricks}
 \usepackage{pstricks-add}
 \usepackage{pgf,tikz}
 \usetikzlibrary{automata}
 \usetikzlibrary{arrows}
 \usepackage{indentfirst}
 %\pagestyle{myheadings}
\usepackage{tabularx} 
\usepackage{enumitem}

\usepackage{amsfonts,amssymb,amsthm,amsmath,eucal,tabu,url}
 \usetikzlibrary{arrows}





\setbeamertemplate{theorems}[numbered]
\theoremstyle{plain}
\newtheorem{thm}{Theorem}[section] %\thethm
%\newtheorem{theorem}[thm]{Theorem}[section]
\newtheorem*{theoremA}{Theorem A}
\newtheorem*{theoremB}{Theorem B}
\newtheorem*{theoremC}{Theorem C}
%\newtheorem{Example}{Example}
\newtheorem{proposition}[thm]{Proposition}
%\newtheorem{corollary}[thm]{Corollary}
\newtheorem{conjecture}[thm]{Conjecture}

\theoremstyle{definition}
\newtheorem{defi}[thm]{Definition}
\newtheorem{remark}[thm]{Remark}
%\newtheorem{example}[thm]{Example}
\newtheorem{claim}[thm]{Claim}
\newtheorem{emma}[thm]{Lemma}
\newtheorem{question}[thm]{Question}
%\newtheorem{problem}[thm]{Problem}
%\newtheorem{fact}[thm]{Fact}
\newtheorem*{notation}{Notation}
\usetheme{Madrid}
\usecolortheme{default}
\newtheorem{conj}{Conjecture}

\newcommand\zbiorn{\{1,2,\ldots,n\}}
\newcommand\B{$[B^i]^n$}
\newcommand\textg{\textcolor{green}}
\newcommand\textb{\textcolor{blue}}
\newcommand{\aalpha}[4]{\alpha_{#1#2}^{#3#4}}
\newcommand\frakm{{\mathfrak m}}
\newcommand\frakP{{\mathfrak P}}
\newcommand\frakL{{\mathfrak L}}
\newcommand\frakI{{\mathfrak I}}
\newcommand\ince{{ (\mathfrak P, \mathfrak L )}}
\newcommand\BB{[B^i]^n\ }
\newcommand{\ppunktP}[2]{P_{#1#2}}
\newcommand{\llinel}[2]{l^{#1}_{#2}}
\newcommand\ffact{Fact\ }
\newcommand\llemma{Lemma\ }
\newcommand\bi{B^i\ }

\author{Filip Zieliński}

\institute{
UKEN Kraków
}

\date{11.01.2024}

\title{On the Shortest Path between two Vertices in Permutation Polytope }

\begin{document}

\titlepage
\newpage
\section{Introduction.}
\begin{frame}{Definitions}
    \begin{definition}\label{def Extreme Points}
    \textbf{Extreme points.} Let $A \subset \mathbb{R}^n $ be a set. A point $a \in A$ is called an \textit{extreme point} of $A$ provided for any two points $b,c \in A$  such that $\frac{b+c}{2} = a$ one must have $b = c = a$. The set of all extreme points of $A$ is denoted ex($A$).  
    \end{definition}
    \pause 
    \begin{definition} \label{def2}
    \textbf{Polytope.} The convex hull of a finite set of points in $ \mathbb{R}^n$ is called a \textit{polytope}.
    \end{definition}
    \begin{definition}
         \textbf{Polyhedron}. Let $c_{1},..., c_{m}$ be vectors from $ \mathbb{R}^n$ and let $\beta_{1}, ... , \beta_{m}$ be real numbers and additionally $\langle \cdot,\cdot \rangle$ be  the standard inner product. The set:
    $$P = \{ x \in \mathbb{R}^n : \langle c_{i},x \rangle \leq \beta_{i},  i=1,..,m\}$$
    is called \textbf{polyhedron.}
    \end{definition}
    An extreme point of a polyhedron is called a \textit{vertex}
   
\end{frame}
\begin{frame}{Extreme Points Theorem}
    \begin{notation}
    An extreme point of a polyhedron is called a \textit{vertex}.
\end{notation}
\begin{Theorem}
    Let $P \subset \mathbb{R}^n$ be a polyhedron:
    $$P = \{ x \in \mathbb{R}^n : \langle c_{i},x \rangle \leq \beta_{i},  i=1,..,m\}$$
    where $c_i \in \mathbb{R}^n$ and $\beta_i \in \mathbb{R}$ for $i=1,..,m$.
    For $u \in P$ let
    $$I(u) = \{i:\langle c_{i},u \rangle  = \beta_i \}$$
    be a set of the inequalities that are active on $u$. Then $u$ is a vertex of $P$ if and only if the set of vectors $\{c_i : i \in I(u) \}$ linearly spans the vector space $\mathbb{R}^n$. In particular if $u$ is a vertex of $P$ the set $I(u)$ contains at least n indices: $|I(u)| \geq n$

\end{Theorem}
\end{frame}
\section{Permutation Polytope}
\begin{frame}{Permutation Polytope} 
    \begin{definition}
    Let us fix a point $x=(\xi_{1},...,\xi_{n}) \in \mathbb{R}^n$. For a permutation $\sigma$ of the set $\{1,...,n\}$ let $\sigma(x)$ be the vector  $y =(\eta_{1},...,\eta_{n})$ where $\eta_{i} = \xi_{\sigma^-1(i)}$
    Let $S_{n}$ be the symmetric group of all permutations of the set $\{1,...,n\}$. Let us define the \textit{permutation polytope} $P(x)$ by
    $$P(x) = conv(\sigma(x) : \sigma \in S_{n})$$
    \end{definition}
    in words: we permute the coordinates of a given vector $x$ in all possible ways and take the convex hull of resulting vectors.
    
\end{frame}
\begin{frame}{Permutohedron}
    \begin{theorem}
        Permutation Polytope $P(x)$ where $x = (1,2,...,n) \in \mathbb{R}^n$ lies in a $n-1$ dimensional affine subspace of euclidan $\mathbb{R}^n$ space.  
    \end{theorem}
    \begin{definition}
        Permutation Polytope $P(x)$ where $x = (1,...,n) \in \mathbb{R}^n$ is called $Permutohedron$.
    \end{definition}
    \begin{theorem}
        Every permutation is a vertex of Permutohedron.
    \end{theorem}

\end{frame}
\begin{frame}{Permutohedron of point $x = (1,2,3)$}
    \centering
    \includegraphics[scale=0.28]{permutohedron-3.png}
\end{frame}
\begin{frame}{Permutohedron of point $x = (1,2,3,4)$}
    \centering
    \includegraphics[scale=0.35]{permutohedron-4.png}
\end{frame}
\begin{frame}{Permutohedron of point $x = (1,2,3,4,5)$}
    \centering
    \includegraphics[scale=0.29]{permutohedron-5.png}
\end{frame}
\section{Edges of Permutohedron.}
\begin{frame}{Facets and Faces of Polyhedron}
    \begin{recall}
        Let $P$ be a polyhedron defined as earlier. For $u \in P$ we define a set inequalities that are active on $u$ as following:
        $$I(u) = \{i:\langle c_{i},u \rangle  = \beta_i \}$$
    \end{recall}
    \begin{definition}
        \textbf{Face.} Let $P \subset \mathbb{R}^n$ be a polyhedron. Let $F \subset P$ be a set of all points $u \in P$ such that some $i_1,...,i_k \in I(u)$ for all u. Set $F$ is called a Face. by dimension of a Face we understand $d = n - dim(span(c_{i_1},..,c_{i_k}))$ 
    \end{definition}
\end{frame} 
\begin{frame}{Facets and Faces of Polyhedron}
    \begin{theorem}
        Every face of Polyhedron is a Polyhedron.
    \end{theorem}
    \textbf{Proposition/Definition}
        Face of $n-1$ dimension is called a \textit{Facet.} \\
        Face of $1$ dimension is called an \textit{Edge.} \\
        Face of $0$ dimension is a \textit{Vertex}. \\
        Every \texit{Edge} contains exactly two vertices.\\
        If there exists and edge between two vertices then the vertices are called \textit{adjacent.} 
\end{frame}
\begin{frame}{Edges of Permutohedron}
    \begin{theorem}
        Let $x_1,x_2 \in \mathbb{R}^n$ be two vertices of Permutohedron associated with permutations $\sigma_1,\sigma_2$. There is an edge between $x_1$ and $x_2$, which means they  are adjacent if and only if there exist transposition (permutation) $\sigma_{i} = (i$ $i+1)$ such that $\sigma_{1} = \sigma_{2} \circ \sigma_i$.
    \end{theorem}
    \pause
    \begin{definition}
        Let $[n] = \{1,...,n\}$ be a set of $n$ first natural numbers.
        Set of points $x \in \mathbb{R}^n$ fulfilling $2^n - 2$ inequalities and one equality:

           $$ \sum_{i \in J} x_i \geq \frac{|J|(|J|+1)}{2}, J \subset [n], J \neq \emptyset, J \neq  [n] $$

           $$ \sum_{i=1}^{n} x_i = \frac{n(n+1)}{2} $$
           is a Permutohedron.
    \end{definition}
\end{frame}
\begin{frame}{Example}
\begin{Example}
$x = (3,4,1,2) \in \mathbb{R}^4$ \\
adjacent vertices to the vertex $x$ are:\\ 
$x_1 = x \circ (1$ $2) = (3,4,2,1)$  \\
$x_2 = x \circ (2$ $3) = (2,4,1,3)$ \\
$x_3 = x \circ (3$ $4) = (4,3,1,2)$ 
\end{Example}  
\end{frame}
\begin{frame}{Permutohedron of point $x = (1,2,3,4)$ }
    \centering
    \includegraphics[scale=0.35]{permutohedron-4.png}
\end{frame}

\section{The Shortest Path between Two Vertices of Permutohedron}
\begin{frame}{Adjacent Vertices as Matrices}
     $\sigma = (2,4,3,1)$, 
     $X^{\sigma} =\begin{bmatrix}
        0 & 0 & 0 & 1 \\ 1 & 0 & 0 & 0 \\ 0 & 0 & 1 & 0 \\ 0 & 1 & 0 & 0
    \end{bmatrix} $ \\ \vspace{0.3cm}
    \pause
    $(2,4,3,1) \circ (1$ $2) = (1,4,3,2)$ \\ \vspace{0.3cm}
    $\begin{bmatrix}
        \mathbf{0} & 0 & 0 & \mathbf{1} \\ \mathbf{1} & 0 & 0 & \mathbf{0} \\ \mathbf{0} & 0 & 1 & \mathbf{0} \\ \mathbf{0} & 1 & 0 & \mathbf{0}
    \end{bmatrix}  \stackrel{\text{(1 2)}}{\longrightarrow} \begin{bmatrix}
        \mathbf{1} & 0 & 0 & \mathbf{0} \\ \mathbf{0} & 0 & 0 & \mathbf{1} \\ \mathbf{0} & 0 & 1 & \mathbf{0} \\ \mathbf{0} & 1 & 0 & \mathbf{0}
    \end{bmatrix}
    $
    
\end{frame}
\begin{frame}{Adjacent Vertices as Matrices}
     $\sigma = (2,4,3,1)$, 
     $X^{\sigma} =\begin{bmatrix}
        0 & 0 & 0 & 1 \\ 1 & 0 & 0 & 0 \\ 0 & 0 & 1 & 0 \\ 0 & 1 & 0 & 0
    \end{bmatrix} $ \\ \vspace{0.3cm}
    $(2,4,3,1) \circ (1$ $2) = (1,4,3,2)$ \\ \vspace{0.3cm}
    $\begin{bmatrix}
        \mathbf{0} &  \mathbf{0} &  \mathbf{0} & \mathbf{1} \\ \mathbf{1} &  \mathbf{0} &  \mathbf{0} & \mathbf{0} \\ 0 & 0 & 1 & 0 \\ 0 & 1 & 0 & 0
    \end{bmatrix}  \stackrel{\text{(1 2)}}{\longrightarrow} \begin{bmatrix}
        \mathbf{1} &  \mathbf{0} &  \mathbf{0} & \mathbf{0} \\ \mathbf{0} &  \mathbf{0} &  \mathbf{0} & \mathbf{1} \\ 0 & 0 & 1 & 0 \\ 0 & 1 & 0 & 0
    \end{bmatrix}
    $
    
\end{frame}
\begin{frame}{Adjacent vertices to vertex $(2,4,3,1)$}
     $(2,4,3,1) \circ (1$ $2) = (1,4,3,2)$ \\ \vspace{0.3cm}
    $\begin{bmatrix}
        \mathbf{0} &  \mathbf{0} &  \mathbf{0} & \mathbf{1} \\ \mathbf{1} &  \mathbf{0} &  \mathbf{0} & \mathbf{0} \\ 0 & 0 & 1 & 0 \\ 0 & 1 & 0 & 0
    \end{bmatrix}  \stackrel{\text{(1 2)}}{\longrightarrow} \begin{bmatrix}
        \mathbf{1} &  \mathbf{0} &  \mathbf{0} & \mathbf{0} \\ \mathbf{0} &  \mathbf{0} &  \mathbf{0} & \mathbf{1} \\ 0 & 0 & 1 & 0 \\ 0 & 1 & 0 & 0
    \end{bmatrix}$  \\ \vspace{0.3cm}
    $(2,4,3,1) \circ (2$ $3) = (3,4,2,1)$ \hspace{1.7cm} $(2,4,3,1) \circ (3$ $4) = (2,3,4,1)$  \\ \vspace{0.3cm}
    $\begin{bmatrix}
       0 &  0 &  0 & 1 \\ \mathbf{1} &  \mathbf{0} &  \mathbf{0} & \mathbf{0} \\\mathbf{0} &  \mathbf{0} &  \mathbf{1} & \mathbf{0}\\ 0 & 1 & 0 & 0
    \end{bmatrix}  \stackrel{\text{(2 3)}}{\longrightarrow} \begin{bmatrix}
        0 & 0 & 0 & 1 \\ \mathbf{0} &  \mathbf{0} &  \mathbf{1} & \mathbf{0} \\\mathbf{1} &  \mathbf{0} &  \mathbf{0} & \mathbf{0} \\ 0 & 1 & 0 & 0
    \end{bmatrix}$\hspace{0.5cm} $\begin{bmatrix}
        0 & 0 & 0 & 1 \\ 1 & 0 & 0 & 0 \\ \mathbf{0} &  \mathbf{0} &  \mathbf{1} & \mathbf{0} \\ \mathbf{0} &  \mathbf{1} &  \mathbf{0} & \mathbf{0} 
    \end{bmatrix} \stackrel{\text{(3 4)}}{\longrightarrow}   \begin{bmatrix}
        0 & 0 & 0 & 1 \\ 1 & 0 & 0 & 0 \\ \mathbf{0} &  \mathbf{1} &  \mathbf{0} & \mathbf{0} \\ \mathbf{0} &  \mathbf{0} &  \mathbf{1} & \mathbf{0} 
    \end{bmatrix}$\vspace{0.3cm}
\end{frame}
\begin{frame}{The Shortest Path between Two Vertices of Permutohedron}
    $\sigma_1 = (2,4,3,1) \hspace{3cm} \sigma_2 = (2,3,1,4)$ \\ \vspace{0.3cm}
    $X^{\sigma_1} = \begin{bmatrix}
        0 & 0 & 0 & 1 \\ 1 & 0 & 0 & 0 \\ \alt<2->{\mathbf{0}}{0} & \alt<2->{\mathbf{0}}{0} & \alt<2->{\mathbf{1}}{1} & \alt<2->{\mathbf{0}}{0} \\ 0 & 1 & 0 & 0
    \end{bmatrix} \hspace{2cm} X^{\sigma_2} = \begin{bmatrix}
        \alt<2->{\mathbf{0}}{0} & \alt<2->{\mathbf{0}}{0} & \alt<2->{\mathbf{1}}{1} & \alt<2->{\mathbf{0}}{0} \\ 1 & 0 & 0 & 0 \\ 0 & 1 & 0 & 0 \\ 0 & 0 & 0 & 1
    \end{bmatrix} $ \\ \vspace{0.3cm}
    \visible<3-> { 
    $\begin{bmatrix}
        0 & 0 & 0 & 1 \\ 1 & 0 & 0 & 0 \\ \textbf{0} & \textbf{0} & \textbf{1} & \textbf{0} \\ 0 & 1 & 0 & 0 
        \end{bmatrix} \stackrel{\text{(2 3)}}{\longrightarrow} \begin{bmatrix}
        0 & 0 & 0 & 1 \\ \textbf{0} & \textbf{0} & \textbf{1} & \textbf{0} \\ 1 & 0 & 0 & 0 \\ 0 & 1 & 0 & 0
        \end{bmatrix} \stackrel{\text{(1 2)}}{\longrightarrow} 
        \begin{bmatrix}
        \textbf{0} & \textbf{0} & \textbf{1} & \textbf{0} \\ 0 & 0 & 0 & 1  \\ 1 & 0 & 0 & 0 \\ 0 & 1 & 0 & 0
        \end{bmatrix}$
    }
    
    
    
\end{frame}
\begin{frame}{The Shortest Path between Two Vertices of Permutohedron}
    $\begin{bmatrix}
        0 & 0 & 1 & 0 \\ 0 & 0 & 0 & 1  \\ \mathbf{1} & \mathbf{0} & \mathbf{0} & \mathbf{0} \\ 0 & 1 & 0 & 0
        \end{bmatrix} \hspace{2cm} X^{\sigma_2} = \begin{bmatrix}
       0 & 0 & 1 & 0 \\ \mathbf{1} & \mathbf{0} & \mathbf{0} & \mathbf{0} \\ 0 & 1 & 0 & 0 \\ 0 & 0 & 0 & 1
    \end{bmatrix}$ \\ \vspace{0.3cm}
    $\begin{bmatrix}
        0 & 0 & 1 & 0 \\ 0 & 0 & 0 & 1  \\ \mathbf{1} & \mathbf{0} & \mathbf{0} & \mathbf{0} \\ 0 & 1 & 0 & 0
        \end{bmatrix} \stackrel{\text{(2 3)}}{\longrightarrow} \begin{bmatrix}
        0 & 0 & 1 & 0 \\ \mathbf{1} & \mathbf{0} & \mathbf{0} & \mathbf{0} \\ 0 & 0 & 0 & 1 \\ 0 & 1 & 0 & 0
        \end{bmatrix}$ \pause $\stackrel{\text{(3 4)}}{\longrightarrow}
        \begin{bmatrix}
       0 & 0 & 1 & 0 \\ 1 & 0 & 0 & 0 \\ 0 & 1 & 0 & 0 \\ 0 & 0 & 0 & 1
    \end{bmatrix} \vspace{0.3cm}
        $
        \pause 
    $$\sigma_1 \circ (2 3) \circ (1 2) \circ (2 3) \circ (3 4) = \sigma_2$$
\end{frame}
\begin{frame}{The Shortest Path between Two Vertices of Permutohedron}
     $\sigma_1 = (2,4,3,1) \hspace{3cm} \sigma_2 = (2,3,1,4)$ \\ \vspace{0.3cm}
    $\begin{bmatrix}
        \mathbf{4: } \\ \mathbf{2: } \\ \mathbf{1: } \\ \mathbf{3: }  
    \end{bmatrix} \begin{bmatrix}
        0 & 0 & 0 & 1 \\ 1 & 0 & 0 & 0 \\ 0 & 0 & 1 & 0 \\ 0 & 1 & 0 & 0
    \end{bmatrix} \hspace{2cm}  \begin{bmatrix}
        \mathbf{1: } \\ \mathbf{2: } \\ \mathbf{3: } \\ \mathbf{4: }  
    \end{bmatrix} \begin{bmatrix}
          0 & 0 & 1 & 0 \\ 1 & 0 & 0 & 0 \\ 0 & 1 & 0 & 0 \\ 0 & 0 & 0 & 1
    \end{bmatrix} $ \\ \vspace{0.3cm}
    \pause
    Problem of finding the shortest path between two vertices is the same as problem of sorting an array swapping only adjacent elements.
\end{frame}
\begin{frame}{Permutohedron vertices coloring}
\centering
    \includegraphics[scale=0.35]{permutohedron-4.png}
\end{frame}
\section{Further Questions}
\begin{frame}{Further Questions and Ideas.}
    \begin{itemize}
        \item[1] Finite Groups found in Permutohedron.
        \item[2] Derangements in Permuatohedron.
        \item[3] Inverse Permutations in Permutohedron. 
        \item[4] Charachteristics of all Faces of Permutohedron.
    \end{itemize}
\end{frame}
\begin{frame}{Literature}
    
\begin{thebibliography}{3}
\bibitem{Convexity}
A. Barvinok \emph{A Course in Convexity} American Mathematical Society, 2002. 
\bibitem{Permutahedron_MIMUW}
G. Lancia and P. Serafini \emph{Compact Extended Linear
Programming Models} \url{http://ndl.ethernet.edu.et/bitstream/123456789/71466/1/77.pdf}
\bibitem{Permutahedron_MIT}
M. Goemans \emph{Smallest Compact Formulation for the Permutahedron} \url{https://math.mit.edu/~goemans/PAPERS/permutahedron.pdf}
\end{thebibliography}
\end{frame}

\end{document}
